
\chapter{NovatelOEM4 GPS Library}\label{welcome-to-novateloem4-gps-librarys-documentation}

\begin{description}
\item[Date]
24 Jul 2018
\item[Version]
0.4
\item[Author]
Bruno Tibério
\item[Contact]
\href{mailto:bruno.tiberio@tecnico.ulisboa.pt}{\nolinkurl{bruno.tiberio@tecnico.ulisboa.pt}}
\end{description}

This module contains a few functions to interact with Novatel OEM4 GPS
devices. Currently only the most important functions and definitions are
configured, but the intention is to make it as much complete as
possible.

A simple example can be run by executing the main function wich creates
a Gps class object and execute the following commands on gps receiver:

\begin{itemize}
\tightlist
\item
  \textbf{begin:} on default port or given port by argv{[}1{]}.
\item
  \textbf{sendUnlogall}
\item
  \textbf{setCom(baud=115200):} changes baudrate to 115200bps
\item
  \textbf{askLog(trigger=2,period=0.1):} ask for log \emph{bestxyz} with
  trigger {ONTIME} and period {0.1}
\item
  wait for 10 seconds
\item
  \textbf{shutdown:} safely disconnects from gps receiver
\end{itemize}

\begin{lstlisting}[language=python,frame=none,backgroundcolor=\color{gray!15},numbers=none]
$python NovatelOEM4.py
\end{lstlisting}


\section{Changelog}
\label{\detokenize{index:changelog}}\begin{quote}\begin{description}
		\item[{version 0.4}] \leavevmode
		Moved from optoparse to argparse module.
		changed Queue to make it compatible with python3 queue. Backwards compatibility is maintained.
		Restructured default location. Moved from Lib folder to base path.
		Moved examples to proper folder. This cause backwards compatibility problems. On import, replace
		\console{import Lib.NovatelOEM4} with simply \console{import NovatelOEM4}
		
		\item[{version 0.3}] \leavevmode
		logging configuration as moved outside module to enable user to use already
		configured logging handler. Check \href{https://docs.python.org/2/howto/logging-cookbook.html\#using-logging-in-multiple-modules{}`}{multimodule logging docs}
		
\end{description}\end{quote}
\begin{quote}\begin{description}
		\item[{version 0.2}] \leavevmode
		data from bestxyz message is now placed into a Queue.Queue() FIFO
		
		\item[{version 0.1}] \leavevmode
		initial release
\end{description}\end{quote}

\section[class NovatelOEM4.Gps]{\prefix{class NovatelOEM4.}\textbf{Gps}\args{sensorName='GPS'}}
Novatel OEM4 GPS library class

This class contents is an approach to create a library for Novatel OEM 4 GPS
\begin{quote}
\begin{description}[style=nextline]
		\item[Parameters] \leavevmode
		\sphinxbfcode{sensorName} (optional) \textendash{} A sensor name if used with multiple devices.
		\item[Returns] An object of class NovatelOEM4.Gps	
\end{description}
\end{quote}

\subsection{askLog}
\noindent \prefix{Gps.}\textbf{askLog}\args{logID=’BESTXYZ’, port=192, trigger=4, period=0, offset=0, hold=0}
Request a log from receiver.
\begin{quote}\begin{description}
		\item[{Parameters}] \leavevmode
		\begin{itemize}
			\item \textbf{logID} \textendash{} log type to request.
			
			\item \textbf{port} \textendash{} port to report log.
			
			\item {} 
			\sphinxstyleliteralstrong{trigger} \textendash{} trigger identifier.
			
			\item \textbf{period} \textendash{} the period of log.
			
			\item \textbf{offset} \textendash{} offset in seconds after period.
			
			\item \textbf{hold} \textendash{} mark log with hold flag or not.
			
		\end{itemize}
		
		\item[{Returns}] \leavevmode
		True or false if command was sucessfull or not.
		
\end{description}\end{quote}

The log request command is defined as:




