%%%%%%%%%%%%%%%%%%%%%%%%%%%%%%%%%%%%%%%%%%%%%%%%%%%%%%%%%%%%%%%%%%%%%%%%
%                                                                      %
%     File: Thesis_01Introduction.tex                                  %
%     Tex Master: Thesis.tex                                           %
%                                                                      %
%     Author: Bruno Tibério                                            %
%     Last modified :  1 Jul 2017                                      %
%                                                                      %
%%%%%%%%%%%%%%%%%%%%%%%%%%%%%%%%%%%%%%%%%%%%%%%%%%%%%%%%%%%%%%%%%%%%%%%%

\chapter{Introduction}
\label{chapter:introduction}

\gls{IST} is currently developing research in autonomous electrical vehicles, namely converting from commercial vehicles with upgraded power management.
This provides a motivating/attracting setup for new students and a testbed for industry solutions and it is the main motivation for this work.

This guide aims to be auxiliary documentation and future memory source as part of the \gls{IST} project named \gls{VIENA} and as well as final report for fellowship BL43/2018. 

The main goal of the project is a conversion of an old electrical car (Fiat Elettra) property of \gls{IST} into an autonomous vehicle as framework for future projects or research in this field and also energy efficiency. Within the conversion, researchers, student and collaborators knowledge acquired during academic cycle is put into test and evaluated in a real situation. 




%%%%%%%%%%%%%%%%%%%%%%%%%%%%%%%%%%%%%%%%%%%%%%%%%%%%%%%%%%%%%%%%%%%%%%%%
\section{Guide Outline}
\label{section:outline} 

This documentation guide is organized in four main chapters, server, main sensors, controllers, miscellaneous.
In the server is provided information about the used hardware, designed pieces, software configuration needed to get started and connections.
In main sensors will be discussed the work done with the current available sensors. New sensor additions should be documented under this chapter.
In controllers will be reported mainly hardware controllers and software necessary to connect, configure  or communicate with it.
Miscellaneous contain other parts that do not fit particularly inside any of those chapters but are necessary or may help in the project.

 
%%%%%%%%%%%%%%%%%%%%%%%%%%%%%%%%%%%%%%%%%%%%%%%%%%%%%%%%%%%%%%%%%%%%%%%%

\section{Contributions}
\label{section:contributions}

Main code contributions are available under Github in 





